\documentclass{beamer}
\usepackage[utf8]{inputenc}
\usepackage{color}
\usepackage{fancyvrb}
\usetheme{Berlin}
\setbeamerfont{caption}{size=\footnotesize}

\title{Résolution de Sudoku en Intelligence Artificielle}
\subtitle{Soutenance IA}
\author{CHAUMONT-FRELET Anatole, LAINE Bastien, POINTIN Damien}
\institute{Génie Mathématique | INSA Rouen}

\begin{document}
    \beamertemplatenavigationsymbolsempty

    \begin{frame}
        \titlepage{}
    \end{frame}

    \section*{Sommaire}
        \begin{frame}
            \begin{columns}[t]
  				\begin{column}{5cm}
  					\tableofcontents[sections={1-4}]
  				\end{column}
  				\begin{column}{5cm}
  					\tableofcontents[sections={5-8}]
  				\end{column}
  			\end{columns}
        \end{frame}

    \section{Introduction}
        \subsection{}
            
\begin{frame}
    \frametitle{Introduction}
    \begin{block}{Sudoku}
    			C’est un jeu (inventé en 1979 par Howard Garns) dont le but est de remplir une grille avec une série de
différents chiffres (éventuellement de lettres ou d’autres symboles). La subtilité du jeu vient des trois règles
suivantes, un même chiffre ne peut se trouver plusieurs fois : dans une même ligne, un même bloc ou une même colonne.
    \end{block}
    \begin{block}{Méthode de résolution}
    		\begin{itemize}
    			\item Algorithme génétique
    			\item Eco-Résolution
    			\item A*
    		\end{itemize}
    \end{block}
\end{frame}


    \section{Présentation méthodes}
        \subsection{Algorithme génétique}
            
\begin{frame}
    \frametitle{Présentation}
    \begin{block}{But}
        \pause
        Imiter le principe de sélection naturelle pour résoudre des problèmes d'optimisation.
    \end{block}
    \pause
    \begin{block}{Avantages}
        \begin{itemize}
            \pause
            \item Bons résultats à des problèmes complexes
            \pause
            \item Inutilité de comprendre parfaitement le problème
            \pause
            \item Simples a implémenter (Difficiles à régler)
        \end{itemize}
    \end{block}
\end{frame}
\begin{frame}
    \frametitle{Idées générales}
    Explication succincte:
    \begin{itemize}
        \pause
        \item Modélisation d'individus-\textbf{génomes} (Ensemble de gènes)
        \pause
        \item Groupement en \textbf{génération}
        \pause
        \item \textbf{Évolution} de la génération
    \end{itemize}
\end{frame}
\begin{frame}
    \frametitle{Déroulement}
    \only<-5>{
        \begin{block}{Étapes clef}
            4 étapes majeures:
            \begin{description}
                \pause[2]
                \item[Évaluation:] Affectation d'un \textbf{fitness} à chaque génome
                \pause[3]
                \item[Sélection:]  \textbf{Choix de représentants} basé sur leur fitness
                \pause[4]
                \item[Croisement:] \textbf{Mélange des gènes} de deux individus
                \pause[5]
                \item[Mutation:] \textbf{Altération de gène} de chaque individu
            \end{description}
        \end{block}
    }
    \pause[6]
    \only<6>{
        \begin{exampleblock}{Schéma}
            \begin{center}
                \includegraphics[scale=0.5]{diagrams/PresAlgoGen.png}
            \end{center}
        \end{exampleblock}
    }
\end{frame}
\begin{frame}
    \frametitle{Exemples de techniques}
    \begin{columns}
        \begin{column}{3cm}
            \begin{block}{Sélection}
                \begin{itemize}
                    \item Rank selection
                    \item Roulette wheel 
                    \item Uniforme
                \end{itemize}
            \end{block}
        \end{column}
        \pause
        \begin{column}{3cm}
            \begin{block}{Croisement}
                \begin{itemize}
                    \item Croisement en N points
                    \item Moyenne
                \end{itemize}
            \end{block}
        \end{column}
        \pause
        \begin{column}{3cm}
            \begin{block}{Mutation}
                \begin{itemize}
                    \item Échange
                    \item Changement 
                \end{itemize}
            \end{block}
        \end{column}
    \end{columns}
\end{frame}

        \subsection{Éco-résolution}
            
\begin{frame}
    \frametitle{Éco-résolution}
    \begin{block}{Principe de résolution}
  		 \begin{itemize}
   			 \item Ne raisonne pas de manière globale mais considère le problème comme des agents devant satisfaire un but
    			 \item Pas d'exploration globale de l'ensemble des états
    			 \item Une perturbation ne modifie que peu le mécanisme de résolution  
         \end{itemize} 
	\end{block}
	\pause
	\begin{alertblock}{Conclusion}
		Pour pouvoir appliquer l'éco-résolution il faut définir un ensemble d'agent dont le but est de tendre vers un état stable.
	\end{alertblock}
\end{frame}

\begin{frame}
    \frametitle{Éco-résolution}
    \begin{block}{Les éco-agents}
		 Ils sont caractérisés de la manière suivante :
  		 \begin{itemize}
   			 \item Un but : relation avec d'autres agents 
    			 \item Un état interne
    			 \item Fonction de perception de gêneur
    			 \item Volonte de satisfaire
    			 \item Obligation de fuir 
    			 \item Actions élementaires qui dépendent de l'application :
    			 faireSatisfaction(), trouverPlacePourFuir() et faireFuite()
         \end{itemize} 
	\end{block}
\end{frame}

\begin{frame}
    \frametitle{Éco-résolution}
    On peut voir l'éco-résolution comme un automate à états finis comme suit: 
    \begin{center}
        \includegraphics[scale=0.5]{images/AutomateEcoResolution.png}
    \end{center}
\end{frame}
        \subsection{A*}
            
\begin{frame}
    \frametitle{A*}

\begin{block}{Principe}

Algorithme de recherche de plus court chemin dans un graphe, d'un nœud initial à un nœud final.

\bigskip


Chaque nœud est pondéré par un coût $F = G + H$.


\bigskip

On utilise deux listes : 

\begin{itemize}
\item la liste fermée, qui contient les nœuds que l'on a déjà étudié

\item la liste ouverte, contenant les nœuds à étudier

\end{itemize}

\end{block}
\end{frame}

\begin{frame}{A*}

\begin{block}{Fonctionnement}

A chaque étape, l'algorithme : 
\begin{itemize}
\item choisit dans la liste ouverte le nœud ``optimal'' possédant la plus petite valeur de F,
\item parcourt ses voisins et les ajoute à la liste ouverte \textit{(si ils n'y sont pas déjà)} 
\item met à jour leur valeurs de G et H.
\item passe le noeud ``optimal'' de la liste ouverte à la liste fermée

\end{itemize}

L'algorithme s'arrête quand il à trouver le nœud final ou qu'il à parcourut tout les chemins possibles sans avoir pu l'atteindre.


\end{block} 

\bigskip


\textit{Le choix des fonctions G et H influe grandement sur l'efficacité de l'algorithme.}

\end{frame}

    \section{Application au problème}
        \subsection{Algorithme génétique}
            \begin{frame}
    \frametitle{Création génération initiale}
    Règle d'or pour l'implémentation:
    \pause
    \begin{center}
        Garder la validité des blocs
    \end{center}
    \pause
    \begin{block}{Mise en place}
        Remplissage des valeurs manquantes de chaque bloc.
    \end{block}
\end{frame}
\begin{frame}
    \frametitle{Évaluation}
    Quelle fonction fitness choisir ?
    \pause
    \begin{block}{Définitions}
        \begin{description}
            \pause
            \item[Score ligne:] nombre d'éléments différents par ligne ($1\leq x \leq taille$)
            \pause
            \item[Score colonne:] nombre d'éléments différents par colonne ($1\leq x \leq taille$)
            \pause
            \item[Total:] Sommes des scores ligne/colonne ($\sum_1^{taille}(sLigne(x)+sCol(x))$)\\
            ($2*taille\leq x\leq 2*taille*taille$)
        \end{description}
    \end{block}
\end{frame}
\begin{frame}
    \frametitle{Sélection}
    Comment sélectionner nos individus?
    \pause[2]
    \begin{block}{But}
        \begin{itemize}
            \pause[3]
            \item Garder les meilleurs éléments
            \pause[4]
            \item Garder diversité
        \end{itemize}
    \end{block}
    \pause
    \begin{block}{Choix}
        \begin{itemize}
            \item Rank-selection \pause[5] $\Rightarrow$ Mauvaise diversité
            \item Uniforme \pause[6] $\Rightarrow$ Perte des bons individus
            \item Roulette wheel
        \end{itemize}
    \end{block}
\end{frame}
\begin{frame}
    \frametitle{Croisement}
    Comment croiser deux individus?
    \pause
    \begin{exampleblock}{Note}
        Il faut conserver la validité des blocs!
    \end{exampleblock}
    \pause
    \begin{block}{Liste de possibilités}
        \begin{itemize}
            \pause
            \item Croisement en N points par blocs
            \pause
            \item Maximisation blocs de ligne/colonne
        \end{itemize}
    \end{block}
\end{frame}
\begin{frame}
    \frametitle{Mutation}
    Quelles mutations nous correspondent le mieux?
    \pause
    \begin{block}{Liste de possibilités}
        \begin{itemize}
            \pause
            \item Échange de deux éléments
            \pause
            \item Remplacement d'un élément (Mais perte de validité des blocs)
        \end{itemize}
    \end{block}
\end{frame}

        \subsection{Éco-résolution}
            \begin{frame}
    \frametitle{Éco-résolution}
    \begin{block}{Définition des éco-agents}
    		\begin{itemize}
    		
    		\item Pour définir l'éco-résolution il nous faut définir une liste d'agent dont le but est de converger vers un état un final (solution du problème), dans notre cas la grille correctement remplie.\\
    		\item Nous avons défini les agents suivants:
    		\begin{itemize}
    			\item Ligne
    			\item Colonne
    			\item Cellule
    		\end{itemize}
    		\item Pas d'agent bloc car ils seront par défaut résolu
    		\end{itemize}

    \end{block}
\end{frame}

\begin{frame}
    \frametitle{Éco-résolution}
    \begin{block}{Les éco-agents Ligne}
    		\begin{itemize}
    			\item Le but d'une ligne est que tous les numéros soient différents
    			\item Gêneurs : cellules dont le numéro est présent plus d'une fois.
    			\item Une ligne ne pourra ni fuir, ni être en recherche de fuite. 
    		\end{itemize}
    \end{block}
    \pause
     \begin{block}{Les éco-agents Colonne}
    		\begin{itemize}
    			\item Le but d'une colonne est que tous les numéros soient différents
    			\item Gêneurs : cellules dont le numéro est présent plus d'une fois.
    			\item Une colonne ne pourra ni fuir, ni être en recherche de fuite. 
    		\end{itemize}
    \end{block}
\end{frame}


\begin{frame}
    \frametitle{Éco-résolution}
    \begin{block}{Les éco-agents Cellule}
    		\begin{itemize}
    			\item Satisfaite si sa ligne et sa colonne sont satisfaites
    			\item Gêneurs : cellules dont le numéro est présent plus d'une fois sur la ligne ou la colonne a laquelle appartient la cellule.
    			\item Pour la place choisie pour une cellule pour fuir nous avons eu plusieurs idées : la plus simple, choisir une cellule de son bloc.
    			\item FaireFuite : deux cellules échangent leur numéro
    		\end{itemize}
    \end{block}
    \pause
    \begin{alertblock}{}
		Concrètement comment cela fonctionne ? 	
	\end{alertblock}
\end{frame}

\begin{frame}
	Pour simplifier l'exemple, nous avons choisi de le faire sur une grille 4*4
    \frametitle{Éco-résolution}
    	\begin{columns}[t]
    \begin{column}{5cm}
    \begin{block}{Grille de depart}
	\begin{center}
         \begin{tabular}{|c|c| |c|c| }
               \hline
               \textbf{1}&&&\\
               \hline
               &&\textbf{2}&\\
               \hline
               \hline
               &&&\\
               \hline
               &\textbf{3}&&\textbf{4}\\                      
            \hline
        \end{tabular}
    \end{center}
    
     \end{block} 

    \end{column}
    
   \pause  
  \begin{column}{5cm}
  \begin{block}{Premiere étape : remplissage aléatoire}
    \begin{center}
         \begin{tabular}{|c|c| |c|c| }
               \hline
               \textbf{1}&2&1&3\\
               \hline
               3&4&\textbf{2}&4\\
               \hline
               \hline
               4&1&2&3\\
               \hline
               2&\textbf{3}&1&\textbf{4}\\                      
            \hline
        \end{tabular}
    \end{center}
  \end{block}   
  \end{column}
 \end{columns}  
\end{frame}

\begin{frame}
    \frametitle{Éco-résolution}
    	\begin{columns}
    \begin{column}{5cm}
    \begin{block}{Resolution colonne 4}
	\begin{center}
         \begin{tabular}{|c|c| |c|c| }
               \hline
               \textbf{1}&2&{\color{green} 1}&{\color{red} 3}\\
               \hline
               3&4&\textbf{2}&4\\
               \hline
               \hline
               4&1&2&3\\
               \hline
               2&\textbf{3}&1&\textbf{4}\\                      
            \hline
        \end{tabular}
    \end{center}
    \pause
    \begin{center}
         \begin{tabular}{|c|c| |c|c| }
               \hline
               \textbf{1}&2& 3&{\color{green} 1}\\
               \hline
               3&4&\textbf{2}&{\color{red} 4}\\
               \hline
               \hline
               4&1&2&3\\
               \hline
               2&\textbf{3}&1&\textbf{4}\\                      
            \hline
        \end{tabular}
    \end{center}
     \end{block} 

    \end{column}
    
   \pause  
  \begin{column}{5cm}
  \begin{block}{Resolution colonne 4 : suite}
	\begin{center}
         \begin{tabular}{|c|c| |c|c| }
               \hline
               \textbf{1}&2&{\color{green} 3}&{\color{red} 4}\\
               \hline
               3&4&\textbf{2}&1\\
               \hline
               \hline
               4&1&2&3\\
               \hline
               2&\textbf{3}&1&\textbf{4}\\                      
            \hline
        \end{tabular}
    \end{center}
    \pause
    \begin{center}
         \begin{tabular}{|c|c| |c|c| }
               \hline
               \textbf{1}&2&4&3\\
               \hline
               3&4&\textbf{2}&1\\
               \hline
               \hline
               4&1&{\color{green} 2}&{\color{red} 3}\\
               \hline
               2&\textbf{3}&1&\textbf{4}\\                      
            \hline
        \end{tabular}
    \end{center}
     \end{block} 
     \end{column}
 \end{columns}  
\end{frame}


\begin{frame}
    \frametitle{Éco-résolution}
    \begin{block}{Grille résolue}
    	On arrive ainsi a la grille résolue :
    	\begin{center}
         \begin{tabular}{|c|c| |c|c| }
               \hline
               \textbf{1}&2&4&3\\
               \hline
               3&4&\textbf{2}&1\\
               \hline
               \hline
               4&1&3&2\\
               \hline
               2&\textbf{3}&1&\textbf{4}\\                      
            \hline
        \end{tabular}
    \end{center}
    \end{block}
\end{frame}
        \subsection{A*}
            
\begin{frame}
    \frametitle{A*}

\begin{block}{Idée générale}

L'idée est de considérer les grilles de Sudoku comme les noeuds d'un graphe. On peut passer d'une grille à une autre en lui ajoutant un chiffre dans une case vide.

\bigskip

Pour appliquer A* au problème du Sudoku, on doit en modifier certains aspects :

\begin{description}
\item[Le graphe :]  le graphe doit être construit au fur et mesure que l'on développe les noeuds.

\item[L'état final :] le noeud final correspond en réalité à n'importe quelle grille complète \textit{(en ayant vérifier qu'elle ne possède pas de doublon)}

\end{description}

\end{block}
\end{frame}

\begin{frame}{A*}
\begin{block}{Construction du graphe}

On doit donc construire de nouvelles grilles quand l'algorithme doit développer un nœud. \textit{On construit autant de grille qu'il y a de valeurs possibles pour les cases vides dans la grille du noeud à développer.}

\end{block}

\begin{alertblock}{Choix des fonctions coût}

On définit bien sur des fonctions coût pour le problème :

\begin{description}
\item[G : ] on choisit le nombre de cases remplies de la grille.

\item[H : ] on prend la somme du nombre de valeurs possibles de chaque cases vides de la grille

\end{description}


\end{alertblock}
\end{frame}

    \section{Modélisation}
        \subsection{Algorithme génétique}
            
\begin{frame}
    \frametitle{Diagramme de classes}
    \begin{block}{Diagramme de classes}
        \begin{center}
            \includegraphics[scale=0.4]{diagrams/AGenClass.png}
        \end{center}
    \end{block}
\end{frame}

        \subsection{Éco-résolution}
            
\begin{frame}
    \frametitle{Éco-résolution}
    
    		\begin{block}{Diagramme de Classe}
    		\begin{center}
     \includegraphics[scale=0.4]{images/ClasseEcoResolution1.png}
    \end{center}
    		    

    		\end{block}

\end{frame}

\begin{frame}
    \frametitle{Éco-résolution}
    		\begin{block}{Diagramme de Classe}
    		\begin{center}
    		
    		     \includegraphics[scale=0.3]{image/ClasseEcoResolution2.png}
    		\end{center}

    		\end{block}

\end{frame}

\begin{frame}
    \frametitle{Éco-résolution}
    		\begin{block}{Diagramme de Séquence}
    		    		\begin{center}

    		     \includegraphics[scale=0.4]{diagrams/sequenceEcoResolution1.png}
    		\end{center}

    		\end{block}

\end{frame}

\begin{frame}
    \frametitle{Éco-résolution}
    		\begin{block}{Diagramme de Séquence}
    		    		\begin{center}

    		     \includegraphics[scale=0.4]{diagrams/sequenceEcoResolution2.png}
    		\end{center}

    		\end{block}

\end{frame}

        \subsection{A*}
            
\begin{frame}
	\frametitle{Cas général - diagramme de classe}

\begin{center}

\includegraphics[scale = 0.4]{diagrams/Astar_classe.png}

\end{center}

\end{frame}


\begin{frame}{A*}

\frametitle{Cas du Sudoku - diagramme de classe}

\begin{center}

\includegraphics[scale = 0.3]{diagrams/Astar_sudoku_classe.png}

\end{center}

\end{frame}

    \section{Résultats/démonstration}
        \subsection{Algorithme génétique}
            
\begin{frame}
    \frametitle{Version finale}
    \begin{block}{Choix}
        \begin{description}
            \pause
            \item[Selection:] \pause Rank selection/Roulette wheel
            \pause
            \item[Croisement:] \pause Separateur de blocs
            \pause
            \item[Mutation:] \pause Échange de case
        \end{description}
    \end{block}
\begin{frame}
    \frametitle{Resultats}
    \begin{columns}
        \begin{column}{3cm}
            \begin{block}{Grille 4x4}
                Avantage:
                \begin{itemize}
                    \pause
                    \item Convergence
                \end{itemize}
                Inconvenient:
                \begin{itemize}
                    \pause
                    \item Hasard
                \end{itemize}
            \end{block}
        \end{column}
        \pause
        \begin{column}{3cm}
            \begin{block}{Grille 9x9}
                Avantage:
                \begin{itemize}
                    \pause
                    \item Optimisation locale
                \end{itemize}
                Inconvenient:
                \begin{itemize}
                    \pause
                    \item Pas optimale
                \end{itemize}
            \end{block}
        \end{column}
    \end{columns}
\end{frame}

        \subsection{Éco-résolution}
            
\begin{frame}
    \frametitle{Éco-résolution}
\end{frame}

        \subsection{A*}
            
\begin{frame}
    \frametitle{A*}
\end{frame}

    \section{Conclusion}
        \subsection{}
            \chapter{Conclusion}
   
Nous avons pu voir que l'algorithme $A^{*}$, la programmation génétique et l'éco-résolution s'appliquait à notre problème de résolution du sudoku. \\
Mais il n'y a bien évidemment pas que ces trois méthodes qui auraient pu nous permettre de résoudre le sudoku. \\
En effet, nous aurions pu voir ce problème comme étant un problème de coloration de graphe. Chaque case étant un nœud du graphe et étant reliée à toutes les cases de sa ligne, de sa colonne et son bloc. Chaque couleur aurait représenté un numéro. \\
Nous aurions également pu voir la résolution de sudoku comme étant un problème de planification 

Enfin, il est évident que la résolution du sudoku par la programmation par contrainte est la résolution la plus facile à coder et la plus efficace.
\end{document}
