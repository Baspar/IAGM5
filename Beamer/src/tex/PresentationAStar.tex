
\begin{frame}
    \frametitle{A*}

\begin{block}{Principe}

Algorithme de recherche de plus court chemin dans un graphe, d'un nœud initial à un nœud final.

\bigskip


Chaque nœud est pondéré par un coût $F = G + H$.


\bigskip

On utilise deux listes : 

\begin{itemize}
\item la liste fermée, qui contient les nœuds que l'on a déjà étudié

\item la liste ouverte, contenant les nœuds à étudier

\end{itemize}

\end{block}
\end{frame}

\begin{frame}{A*}

\begin{block}{Fonctionnement}

A chaque étape, l'algorithme : 
\begin{itemize}
\item choisit dans la liste ouverte le nœud ``optimal'' possédant la plus petite valeur de F,
\item parcourt ses voisins et les ajoute à la liste ouverte \textit{(si ils n'y sont pas déjà)} 
\item met à jour leur valeurs de G et H.
\item passe le noeud ``optimal'' de la liste ouverte à la liste fermée

\end{itemize}

L'algorithme s'arrête quand il à trouver le nœud final ou qu'il à parcourut tout les chemins possibles sans avoir pu l'atteindre.


\end{block} 

\bigskip


\textit{Le choix des fonctions G et H influe grandement sur l'efficacité de l'algorithme.}

\end{frame}
