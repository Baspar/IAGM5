\begin{frame}
    \frametitle{Éco-résolution}
    \begin{block}{Définition des éco-agents}
    		\begin{itemize}
    		
    		\item Pour définir l'éco-résolution il nous faut définir une liste d'agent dont le but est de converger vers un état un final (solution du problème), dans notre cas la grille correctement remplie.\\
    		\item Nous avons défini les agents suivants:
    		\begin{itemize}
    			\item Ligne
    			\item Colonne
    			\item Cellule
    		\end{itemize}
    		\item Pas d'agent bloc car ils seront par défaut résolu
    		\end{itemize}

    \end{block}
\end{frame}

\begin{frame}
    \frametitle{Éco-résolution}
    \begin{block}{Les éco-agents Ligne}
    		\begin{itemize}
    			\item Le but d'une ligne est que tous les numéros soient différents
    			\item Gêneurs : cellules dont le numéro est présent plus d'une fois.
    			\item Une ligne ne pourra ni fuir, ni être en recherche de fuite. 
    		\end{itemize}
    \end{block}
    \pause
     \begin{block}{Les éco-agents Colonne}
    		\begin{itemize}
    			\item Le but d'une colonne est que tous les numéros soient différents
    			\item Gêneurs : cellules dont le numéro est présent plus d'une fois.
    			\item Une colonne ne pourra ni fuir, ni être en recherche de fuite. 
    		\end{itemize}
    \end{block}
\end{frame}


\begin{frame}
    \frametitle{Éco-résolution}
    \begin{block}{Les éco-agents Cellule}
    		\begin{itemize}
    			\item Satisfaite si sa ligne et sa colonne sont satisfaites
    			\item Gêneurs : cellules dont le numéro est présent plus d'une fois sur la ligne ou la colonne a laquelle appartient la cellule.
    			\item Pour la place choisie pour une cellule pour fuir nous avons eu plusieurs idées : la plus simple, choisir une cellule de son bloc.
    			\item FaireFuite : deux cellules échangent leur numéro
    		\end{itemize}
    \end{block}
    \pause
    \begin{alertblock}{}
		Concrètement comment cela fonctionne ? 	
	\end{alertblock}
\end{frame}

\begin{frame}
	Pour simplifier l'exemple, nous avons choisi de le faire sur une grille 4*4
    \frametitle{Éco-résolution}
    	\begin{columns}[t]
    \begin{column}{5cm}
    \begin{block}{Grille de depart}
	\begin{center}
         \begin{tabular}{|c|c| |c|c| }
               \hline
               \textbf{1}&&&\\
               \hline
               &&\textbf{2}&\\
               \hline
               \hline
               &&&\\
               \hline
               &\textbf{3}&&\textbf{4}\\                      
            \hline
        \end{tabular}
    \end{center}
    
     \end{block} 

    \end{column}
    
   \pause  
  \begin{column}{5cm}
  \begin{block}{Premiere étape : remplissage aléatoire}
    \begin{center}
         \begin{tabular}{|c|c| |c|c| }
               \hline
               \textbf{1}&2&1&3\\
               \hline
               3&4&\textbf{2}&4\\
               \hline
               \hline
               4&1&2&3\\
               \hline
               2&\textbf{3}&1&\textbf{4}\\                      
            \hline
        \end{tabular}
    \end{center}
  \end{block}   
  \end{column}
 \end{columns}  
\end{frame}

\begin{frame}
    \frametitle{Éco-résolution}
    	\begin{columns}
    \begin{column}{5cm}
    \begin{block}{Resolution colonne 4}
	\begin{center}
         \begin{tabular}{|c|c| |c|c| }
               \hline
               \textbf{1}&2&{\color{green} 1}&{\color{red} 3}\\
               \hline
               3&4&\textbf{2}&4\\
               \hline
               \hline
               4&1&2&3\\
               \hline
               2&\textbf{3}&1&\textbf{4}\\                      
            \hline
        \end{tabular}
    \end{center}
    \pause
    \begin{center}
         \begin{tabular}{|c|c| |c|c| }
               \hline
               \textbf{1}&2& 3&{\color{green} 1}\\
               \hline
               3&4&\textbf{2}&{\color{red} 4}\\
               \hline
               \hline
               4&1&2&3\\
               \hline
               2&\textbf{3}&1&\textbf{4}\\                      
            \hline
        \end{tabular}
    \end{center}
     \end{block} 

    \end{column}
    
   \pause  
  \begin{column}{5cm}
  \begin{block}{Resolution colonne 4 : suite}
	\begin{center}
         \begin{tabular}{|c|c| |c|c| }
               \hline
               \textbf{1}&2&{\color{green} 3}&{\color{red} 4}\\
               \hline
               3&4&\textbf{2}&1\\
               \hline
               \hline
               4&1&2&3\\
               \hline
               2&\textbf{3}&1&\textbf{4}\\                      
            \hline
        \end{tabular}
    \end{center}
    \pause
    \begin{center}
         \begin{tabular}{|c|c| |c|c| }
               \hline
               \textbf{1}&2&4&3\\
               \hline
               3&4&\textbf{2}&1\\
               \hline
               \hline
               4&1&{\color{green} 2}&{\color{red} 3}\\
               \hline
               2&\textbf{3}&1&\textbf{4}\\                      
            \hline
        \end{tabular}
    \end{center}
     \end{block} 
     \end{column}
 \end{columns}  
\end{frame}


\begin{frame}
    \frametitle{Éco-résolution}
    \begin{block}{Grille résolue}
    	On arrive ainsi a la grille résolue :
    	\begin{center}
         \begin{tabular}{|c|c| |c|c| }
               \hline
               \textbf{1}&2&4&3\\
               \hline
               3&4&\textbf{2}&1\\
               \hline
               \hline
               4&1&3&2\\
               \hline
               2&\textbf{3}&1&\textbf{4}\\                      
            \hline
        \end{tabular}
    \end{center}
    \end{block}
\end{frame}