\begin{frame}
    \frametitle{Éco-résolution - Version 0}
    \begin{block}{Choix d'implémentation}
    		Version totalement naïve :
    		\begin{itemize}
    			\item Trouver place pour fuir : une case aléatoire dans le bloc
    			\item Choix éco-agent : aléatoire également
    		\end{itemize}
    \end{block}
    \pause
    \begin{alertblock}{Résultat}
    		Fonctionne pour un sudoku 4*4 mais pas pour un 9*9
    \end{alertblock}
\end{frame}


\begin{frame}
    \frametitle{Éco-résolution - Version 1}
    \begin{block}{Choix d'implémentation}
    		Améliorations effectuées
    		\begin{itemize}
    			\item EssayerSatisfaire pour l'écoAgent suppression de la boucle sur les gêneurs
    			\item Trouver place pour fuir : une case aléatoire parmi les cellules non satisfaites du bloc avec une proba de 95%
    			\item Choix éco-agent : parmi les éco-Agents où il y a le plus d'erreur 
    		\end{itemize}
    \end{block}
    \pause
    \begin{alertblock}{Résultat}
    		Fonctionne pour une grille facile de sudoku 9*9 mais lent (40sec)
    \end{alertblock}
\end{frame}

\begin{frame}
    \frametitle{Éco-résolution - Version 2}
    \begin{block}{Choix d'implémentation}
    		Amélioration effectuée
    		\begin{itemize}
    			\item Trouver place pour fuir : ajout d'une contrainte (l'éco-Agent qui essaie d'être satisfait) donc choix d'une case non satisfaite dans le bloc qui n'est pas sur la ligne(ou colonne)
    		\end{itemize}
    \end{block}
    \pause
    \begin{alertblock}{Résultat}
    		Fonctionne pour une grille facile (1,7 sec) et moyenne (51 sec)
    \end{alertblock}
    \pause
    \begin{exampleblock}{Autre test}
		Choisir l'éco-Agent avec le plus gros score n'étant pas résolu
	\end{exampleblock}
\end{frame}


\begin{frame}
    \frametitle{Éco-résolution - Version 3}
    \begin{block}{Choix d'implémentation}
    		Amélioration effectuée
    		\begin{itemize}
    			\item Trouver place pour fuir : modification contrainte, choix aléatoire parmi les cellules dont le numéro apparaît le moins de fois dans la contrainte
    		\end{itemize}
    \end{block}
    \pause
    \begin{alertblock}{Résultat}
    		Fonctionne pour une grille facile (0,2 sec), moyenne (1.425 sec), difficile (14.7 sec) et diabolique (51.95 sec)
    \end{alertblock}
\end{frame}


\begin{frame}
    \frametitle{Éco-résolution - Version 2}
    \begin{exampleblock}{Autres tests}
   		 \begin{itemize}
    				\item Choisir parmi toutes les cellules et non pas seulement celles qui sont non-satisfaites
    		 \pause
    				\item Ajouter une mémoire aux cellules pour pas qu'elles aient deux fois le même numéro que pendant les deux tours précédents 
		 \pause
				\item Combiner le test précédent avec la méthode la plus efficace précédente
		 \end{itemize}
	\end{exampleblock}
	\pause
	\begin{alertblock}{Résultat}
		Fonctionne mais n'améliore pas le résultat, beaucoup plus lent avec la mémoire (50sec pour une grille facile).
    \end{alertblock}
\end{frame}