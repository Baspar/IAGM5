\input{ressources/tex/ExampleTex.tex}

\chapter{Présentation du problème}
  
Dans cette partie nous présenterons brievement les règles du Sudoku ainsi que différentes méthode de résolutions.

\textcolor{BurntOrange}{éventuellement les structures de données utilisées ?}

\section{Règles du Sudoku}

\textcolor{BurntOrange}{Wikipediaaaaaaaaaaaaaaaaaaaaaaaaaaaaaaaaaaaaaaaaaaaaaaaaaaaaaaaaaa}


Le termes Sudoku provient de l'abbrévation de la phrase japonaise : \textit{"chiffre limité à un seul"} ou \textit{"chiffre unique"}

C'est un jeu \textit{(inventé en 1979 par Howard Garns)} dont le but est de remplir une grille avec une série de \textbf{différents} chiffres \textit{(éventuellement de lettres ou d'autres symboles)}. La subtilité du jeu vient des trois règles suivantes, un même chiffre ne peut se trouver plusieurs fois :

\begin{itemize}

\item dans une même ligne
\item dans une même colonne
\item dans une même sous-grille

\end{itemize}

Les sous-grille sont définies selon les dimension de la grille principale, le Sudoku le plus répendu utilise les chiffres de 1 à 9, une grille de 9 par 9 et donc des sous-grilles de 3 par 3. Il exite bien différentes variantes avec des dimensions ou des symboles différents, ou même avec des sous-grilles de forme irrégulière.

Le jeu commence avec une grille vide dans laquelle sont déjà disposés quelques symboles, permettant la résolution \textbf{progressive} du problème.




\section{Méthodes de résolution}
