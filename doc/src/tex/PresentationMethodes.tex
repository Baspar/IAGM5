\chapter{Méthodes}
    \section{A*}
    \section{Algorithme génétique}
        \subsection{Généralité}
            Les algorithmes génétiques sont des algorithme basés sur là représentation de l'évolution dans la génétique.\\
            En premier lieu, l'algorithme va créer un ensemble d'individu, au propriétés aléatoires.
            Par la suite, l'algorithme va dérouler 3 phases bien distinctes, jusqu'à arriver à un individu qui répondra complètement au problème: 
            \begin{itemize}
                \item \textbf{Sélection}: parmi tous les individus présents, seuls les plus pertinents (Ceux dont la fonction de \textbf{fitness} sera la plus grande) seront gardés.
                \item \textbf{Croisement}: une fois les individus choisis, il faut les utiliser Pour recréer un nouvel ensemble d'individu plus aptes, \textbf{réutilisant les gènes des parents} présents.\\
                    La manière dont le croisement est fait dépend du problème.
                \item \textbf{Mutation}: Une fois notre nouvelle espèce créée, chaque individu se verra ajouter \textbf{une part d'aléatoire} dans son génome.
            \end{itemize}
            Voyons en détails ces trois étapes.
        \subsection{Sélection}
            Comme dit précédemment, une sélection se fait sur les individus présents.\\
            Au même titre que pour $A^*$, il nous faut définir une échelle qui nous permettra de comparer les individus entre eux pour déterminer quels sont ceux les plus à même d'apporter des bribes de réponse à notre problème.\\
            Nous allons donc définir pour chaque problème une \textbf{fonction fitness} qui prendra le génome de l'individu, et qui en donnera un score, totalement indépendant d'autre facteur.\\
            Dans notre cas, nous aborderons le sujet plus tard, mais cela pourrai par exemple être le nombre de conflit présents sur une grille donnée (Auquel cas il faudrait non plus maximiser cette fonction, mais la minimiser)
        \subsection{Croisement}
            Les croisement a pour intérêt de créer une nouvelle population, qui ne sera pas aléatoire (contrairement à la première), mais qui utilisera les attributs de ces meilleurs prédécesseurs.\\
            la manière la plus simple de voir une telle chose, est en représentant chacun des génomes par un tableau d'entier.\\
            Avec deux tels parents, on peut facilement créer un mélange entre les deux (En prenant par exemple la moitié de chaque tableaux, ce qui nous donnera 2 enfants).\\
            Le piège est malgré tout de croiser intelligemment.\\
            En effet, si pour notre problème de Sudoku, nous prenions 1 case sur 2 pour faire le croisement, cela n'assurera en rien d'avoir des enfants dont le score risque de monter.
        \subsection{Mutation}
            Le dernier stade, et le plus ``crucial'' reste la mutation.\\
            En effet, même après avoir croisé l'ensemble des gènes des parents, ils se peut qu'il manquerait un gène pour résoudre le problème.\\
            C'est la raison pour laquelle chacun de nos enfants se verra modifié de manière aléatoire un ou plusieurs de ces gènes.
    \section{Résolution humain}
    \section{Eco-résolution}
    \subsection{Principe de résolution}
        L'éco-résolution est une méthode de résolution de problème qui prend le contre-pied des techniques classiques. Plutôt que de résonner que manière globale et définir des méthodes de résolution, l'éco-résolution préfère considérer le problème comme des agents en interactions devant satisfaire un but. \\ \\
        Il faut donc pour pouvoir appliquer l'éco-résolution définir un ensemble d'agents dont le but est de tendre vers un état stable (solution du problème). Chaque agent répond à deux principes importants: autonomie et localité. C'est à dire que chaque agent agit de manière locale mais aussi en fonction des interactions qu'il a avec les différents agents avec lesquels il est en relation.
        Les particularités de l'éco-résolution sont les suivantes:
        \begin{itemize}
        \item Pas d'exploration globale de l'ensemble des états. Seuls les états des différents agents sont pris en compte. 
        \item Résiste très bien au bruit: une perturbation ne modifie que peu le mécanisme de résolution, en effet c'est une donnée normale dans le principe de résolution.
        \item De ce fait permet de résoudre des problèmes de grand taille. 
        \end{itemize}
    
        \subsection{L'eco-resolution vu comme un automate}
        \textbf{Il faut mettre le schéma bizarre la }
