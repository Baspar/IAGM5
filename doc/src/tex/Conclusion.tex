\chapter{Conclusion}
   
Nous avons pu voir que l'algorithme $A^{*}$, la programmation génétique et l'éco-résolution s'appliquait à notre problème de résolution du sudoku. \\
Mais il n'y a bien évidemment pas que ces trois méthodes qui auraient pu nous permettre de résoudre le sudoku. \\
En effet, nous aurions pu voir ce problème comme étant un problème de coloration de graphe. Chaque case étant un nœud du graphe et étant reliée à toutes les cases de sa ligne, de sa colonne et son bloc. Chaque couleur aurait représenté un numéro. \\
Nous aurions également pu voir la résolution de sudoku comme étant un problème de planification 

Enfin, il est évident que la résolution du sudoku par la programmation par contrainte est la résolution la plus facile à coder et la plus efficace.